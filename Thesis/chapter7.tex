\chapter{Figure and Equations formatting}
\vspace{-1.4em}

\section{cut outs}

\subsection{wireless comms chapter 2}
Another factor which has a strong influence on the fading channel is the Doppler Effect. The Doppler Effect 
occurs when a receiver or transmitter moves location as seen in Figure~\ref{fig:Doppler}. 
\begin{figure}[ht]
 \begin{center}
 %\includegraphics[width=12.5cm]{images/q4a}
%trim option's parameter order: left bottom right top
 \includegraphics[trim = 26.9mm 227mm 66.4mm 26.4mm, clip, width=12.5cm]{images/Doppler}
 \end{center} 
 \vspace{-1cm}
 \caption{Multipath Induced Doppler Effect~\cite{Goldsmith_2005}}
 \label{fig:Doppler}
\end{figure}
The movement of the receiver or transmitter affects the relative locations of reflectors in the wireless 
environment. As stated earlier the reflectors allow alternate paths for signal propagation to the receiver. 
As the relative location of objects change when the receiver or transmitter move, then a different set of 
multipath signals are experienced at the receiver. This phenomenon gives rise to the apparent time-varying 
nature of the channel and leads to rapid changes in the signal strength over small distances.

\subsection{system model chapter 3}
Nakagami-m Fading
The fading amplitude ($H$) in a wireless channel is generally modelled as Nakagami-m distributed (REFERENCE). As m approaches infinity, the Nakagami distribution approaches one and this scenario represents the case of no fading. Setting the value of m to one, results in the well known case of Rayleigh fading and this type of is used ubiquitously throughout this thesis. Thus, it is evident that different values for m result in different fading distributions which are relevant to different transmission environments.





\section{Attenuation in the Wireless Medium}

example code

\textbf{matlab figures}
\begin{figure}[ht]
 \begin{center}
  %\includegraphics[width=12.5cm]{images/q4a}
%trim option's parameter order: left bottom right top
  \includegraphics[trim = 32.5mm 94mm 31mm 94mm, clip, width=12.5cm]{images/q2iii}
  \end{center}

 \vspace{-1cm}
 %\includegraphics[width=12.5cm]{images/propagation_effects}
 \caption{Path loss, shadowing, and multipath versus distance~\cite{Goldsmith_2005}}
 \label{fig:prop_effects}
\end{figure}


Italics
\textit{large-scale propagation effects} 
\hilight{($100$-$1000$ m)}


Subequations
\begin{align}
 y(x) &= \sum_0^{\infty}a_nx^{n+r}     \label{eqn:frobenius 6}\\
 \implies y'(x) &= \sum_0^{\infty}(n+r)a_nx^{n+r - 1}	\label{eqn: frobenius 7}  \\
 \implies y''(x) &= \sum_0^{\infty}(n+r-1)(n+r)a_nx^{n+r - 2}	\label{eqn: frobenius 8} 
\end{align}


matlab code put in here
\begin{verbatim*}
 
\end{verbatim*}
